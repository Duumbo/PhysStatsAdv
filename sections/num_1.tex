\section{Section I} \label{sec: sec_1_1}
\blindtext[4]\cite{texbook}

\subsection{Sous-section I} \label{subsec: sub_1_1}
\begin{wrapfigure}{r}{0.4\textwidth}
    \centering
    \includegraphics[scale=0.3]{figs/Hamilton_portrait.png}
    \caption{M. Hamilton.\cite{texbook}}
    \label{fig: hamilton_portrait}
\end{wrapfigure}
\blindtext[4]\cite{latex:companion}

\subsubsection{Sous-sous-section I} \label{subsubsec: subsub_1_1}
\blindtext\cite{latex2e}

\begin{table}[h!]
    \centering
    \begin{tabular}{cccc} \toprule
        $V_1$ (V) & $V_2$ (V) & $V_3$ (V) & $V_4$ (V) \\\midrule\midrule
        $290 \pm 4$ & $295 \pm 4$ & $304 \pm 4$ & $310 \pm 4$ \\
        $290 \pm 4$ & $295 \pm 4$ & $304 \pm 4$ & $310 \pm 4$ \\
        \bottomrule
    \end{tabular}
    \caption{Caption, caption. Caption caption captioncaptioncaption caption caption
    .}
    \label{table: tableau_1.1.1}
\end{table}

\blindtext[3]

\begin{figure}[h!]
    \begin{center}
        \includegraphics[width=0.6\textwidth]{figs/sphere.pdf}
    \end{center}
    \caption{Une sphère molle?}
    \label{fig: oui_une_sphere}
\end{figure}

\noindent
Pour commencer cette partie du problème, on doit redéfinir les relations de conservation de la quantité de mouvement parallèle et de l'énergie. Soit
\begin{align*}
    \Vec{p}:
    \left\{
    \begin{array}{c}
        p\sin i = p'\sin\alpha \\\\
        p'\sin\alpha = p''\sin\beta
    \end{array}
    \right.
    \betspace
    E:
    \left\{
    \begin{array}{c}
        \frac{p^2}{2m} = \frac{p'^2}{2m} + V_0 \\\\
        \frac{p'^2}{2m} + V_0 = \frac{p''^2}{2m}
    \end{array}
    \right.,
\end{align*}

où l'on remarque directement que $p^2 = p''^2$ et en substituant cette relation dans les équations de quantité de mouvement on trouve que
\begin{align*}
    p\sin i = p''\sin\beta\implies\beta = i.
\end{align*}
Cette déduction nous permettra de trouver la relation entre les
angles $\theta, \alpha, i$. En effet, en observant le schéma de
la figure, on voit nécessairement que
\begin{align*}
    \underbrace{\pi/2 = \delta + \beta + \theta}_A,\;\;\; \underbrace{\pi = i + (\pi - 2\alpha) + \varphi}_B\betspace \underbrace{\pi = \varphi + \pi/2 + \delta}_C.
\end{align*}
L'équation $B$ nous offre une expression pour l'angle $\varphi = 2\alpha - i$ tandis que l'équation $A$ nous offre une expression pour $\delta = \pi/2 - \beta - \theta$. En substituant ces derniers résultats dans l'équation $C$, on aura alors
\begin{align*}
    \pi = (2\alpha - i) + \pi/2 + (\pi/2 - \beta - \theta)\implies\theta = \beta + i - 2\alpha,
\end{align*}

qui lorsque nous utilisons le fait que $i = \beta$, alors nous avons
plutôt
\begin{align*}
    \theta = 2(i - \alpha).
\end{align*}
À partir d'ici, nous souhaitons retrouver l'équation et pour ce
faire, on manipulera notre dernier résultat
\begin{align}
    \boxed{
        \cos\left(\frac{\theta}{2}\right) = \cos(i)\cos(\alpha) +
        \sin(i)\sin(\alpha),
    }
    \label{eq: test_eq}
\end{align}
où ici nos précédente conclusions peuvent être réutilisées telles
que l'équation ainsi que le fait
$\sin\alpha = b/a$. \\

\blindtext[2]

\clearpage

