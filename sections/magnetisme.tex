% Removing titlerule from section: Introduction
\titleformat{\section}{
\normalfont\Large\bfseries{
  \color{\MasterColor}\titlerule[0.0pt]
  }
}{\thesection}{1em}{}

\section{Magnétisme} \label{sec: intro}

La magnétostatique débute avec la loi de Biot-Savard:
\begin{align}
    \vb{B}(\vb{r})&=\frac1c \int\vb{J}(\vb{r})\times\frac{\vb{r}-\vb{r}'}{\norm{\vb{r}-\vb{r}'}^3}\dd[3]{r'}.
\end{align}
Ce sera notre point de départ pour expliquer les phénomènes magnétiques classiquement
pour commencer. Notre but est de justifier l'Hamiltonien de Pauli, pour ce faire,
nous aurons besoin de justifier l'équation de Dirac.
On peut calculer que

\begin{align}
    \frac{\vb{r}-\vb{r'}}{\norm{\vb{r}-\vb{r}'}^3}&=-\nabla\qty(\frac{1}{\norm{\vb{r}-\vb{r'}}}).
\end{align}
Nous allons utiliser la notation que $\nabla$ représente la dérivée par rapport
au vecteur $\vb{r}$, nous pourrions aussi utiliser la notation $\nabla_{\vb{r}}$.
Lorsque nous voulons utiliser la dérivée par rapport au vecteur $\vb{r'}$, nous
utilisons $\nabla'$.
\begin{align}
    \vb{B}(\vb{r})&=\frac1c\nabla\times\int\frac{\vb{J}(\vb{r'})}{\norm{\vb{r}-\vb{r'}}}\dd[3]{r'}.
\end{align}
On obtient cette équation grâce aux propriétés du gradient:
\begin{align}
    \nabla\times(f\vb{J})&=\nabla f\times\vb{J}+f\nabla\times\vb{J},
\end{align}
en prenant $f=\frac1{\norm{\vb{r}-\vb{r'}}}$.
\begin{align}
    \nabla\times(f\vb{J})&=\nabla\qty(\frac1{\norm{\vb{r}-\vb{r'}}})\times\vb{J}
        +\frac1{\norm{\vb{r}-\vb{r'}}}\nabla\times\vb{J}(\vb{r'})
\end{align}
Le rotationnel d'une quantité ne dépendant pas explicitement de $\vb{r}$ est
le vecteur nul. On voit donc émerger le potentiel vecteur:
\begin{align}
    \vb{B}&=\nabla\times\vb{A}\\
    \vb{A}&=\frac1c\int\frac{\vb{J}(\vb{r'})}{\norm{\vb{r}-\vb{r'}}}\dd[3]{r'}.
\end{align}
Lorsqu'on parle de potentiel vecteur, il est important de bien comprendre la notion
de choix de jauge. On peut se faire une image de ce que le choix de jauge est
en pensant au choix du zéro de l'énergie potentielle. Le potentiel vecteur est
lié à la densité de courant pondérée par la distance.

\subsection{Unités CGS} % (fold)
\label{sub:Unités CGS}
Les unités CGS sont définies avec le centimètre, le gramme et la seconde comme
unités de base, ce qui est différent de SI où l'on utilise le mètre, le kilogramme
et la seconde. Les équations de Maxwell en SI sont

\begin{align}
    \nabla\cdot\vb{B}&=0\\
    \nabla\cdot\vb{E}&=\frac\rho\epsilon_0\\
    \nabla\times\vb{B}&=\mu_0\qty(\epsilon_0\pdv{\vb{E}}{t}+\vb{J})\\
    \nabla\times\vb{E}&=-\pdv{\vb{B}}{t},
\end{align}

En CGS, elles s'écrivent plustot

\begin{align}
    \nabla\cdot\vb{B}&=0\\
    \nabla\cdot\vb{E}&=4\pi\rho\\
    \nabla\times\vb{B}&=\frac1c\qty(\pdv{\vb{E}}{t}+4\pi\vb{J})\\
    \nabla\times\vb{E}&=-\frac1c\pdv{\vb{B}}{t}
\end{align}

% subsection Unités CGS (end)

\subsection{Expension multipolaire} % (fold)
\label{sub:Expension multipolaire}

On peut faire l'expension de

\begin{align}
    \frac1{\norm{\vb{r}-\vb{r'}}}=\frac1{\norm{r}}+\frac{\vb{r}\cdot\vb{r'}}{\norm{\vb{r}}^3}+\cdots
\end{align}
où cette expression est une bonne approximation pour $\norm{\vb{r'}}\ll\norm{\vb{r}}$.
On peut réécrire le potentiel vecteur

\begin{align}
    \vb{A}&=\frac1c\frac1{\norm{r}}\int\vb{J}(\vb{r'})\dd[3]{r}
    +\frac1c\frac1{\norm{r}^3}\int(\vb{r}\cdot\vb{r'})\vb{J}(\vb{r'})\dd[3]{r'}
    +\cdots
\end{align}

On définit donc
\begin{align}
    \vb{A}_{\rm dip.}&=-\frac1c\frac1{\norm{r}^3}\sum_i\hat{n}_i\frac12\qty[
    \vb{r}\times\int(\vb{r'}\times\vb{J})\dd[3]{r'}]_i\\
    \vb{M}(\vb{r}&=\frac1{2c}\vb{r}\times\vb{J}(\vb{r}),
\end{align}
où $\vb{M}$ est la densité de moment magnétique dipolaire.
\begin{align}
    \vb{A}_{\rm dip.}&=-\frac1{\norm{\vb{r}}^3}\vb{r}\times
    \int\vb{M}(\vb{r'})\dd[3]{r'}\\
                     &=-\frac{\hat{r}\times\vb{m}}{r^2},
\end{align}
où $\vb{m}$ est le moment magnétique dipolaire. On peut écrire le champ magnétique
lié au potentiel vecteur dipolaire que nous avons écrit

\begin{align}
    \vb{B}_{\rm dip.}(\vb{r})&=\nabla\times\vb{A}_{\rm dip.}(\vb{r})\\
                             &=\frac{3\hat{r}(\hat{r}\cdot\vb{m})-\vb{m}}{\norm{r}^3}
\end{align}

% Demo
Soit prenons le rotationnel du potentiel vecteur et utilisons l'identité de Jacobi
\begin{align}
    \nabla\times\vb{A}_{\rm dip.}
    &=-\nabla\times\qty(\frac{\hat{r}\times\vb{m}}{r^2})\\
    &=-\qty(\vb{m}\times\qty(\nabla\times\frac{\hat{r}}{r^2})
    -\hat{r}\times\qty(\nabla\times\frac{\vb{m}}{r^2})).
\end{align}
Or, $\frac{\hat{r}}{r^2}$ est irrotationnel, donc
\begin{align}
    \nabla\times\vb{A}_{\rm dip.}
    &=\hat{r}\times\qty(\nabla\times\frac{\vb{m}}{r^2})
\end{align}
:(


qui est une propriété à démontrer pour le prochain cours. On peut considérer le
solide comme un ensemble de petites boucles de courant, ce qui ferait émerger
le magnétisme

\begin{align}
    \vb{m}&=\frac{I}{2c}\int\vb{r}\times\dd{\boldsymbol{\ell}}\\
    \vb{J}(\vb{r})&=\sum_iq_i\vb{v}_i\delta(\vb{r}-\vb{r}_i\\
    \vb{m}&=\sum_i\gamma_i\vb{L}_i\\
    \gamma_i&=\frac{q_i}{2M_ir},
\end{align}
qui est l'image classique du magnétisme microscopique.

% subsection Expension multipolaire (end)

\subsection{Force de Lorentz} % (fold)
\label{sub:Force de Lorentx}

La force de Lorentz est

\begin{align}
    \vb{F}&=q\qty(\vb{E}+\frac1c\vb{v}\times\vb{B}).
\end{align}

On peut aussi exprimer le lagrangien non-relativiste

\begin{align}
    \mathscr{L}&=\frac12mv^2-V\\
    V&=q\phi(\vb{r})-\frac qc\vb{v}\cdot\vb{A}(\vb{r},t),
\end{align}

où $\phi(\vb{r})$ est le potentiel électrostatique tel que $\vb{E}=-\nabla\phi$.
On peut montrer que les équations différentielles sont équivalentes en  utilisant
les équations d'Euler-Lagrange.

\begin{align}
    \dv{t} \qty(\pdv{\mathscr{L}}{\dot{q_i}})-\pdv{\mathscr{L}}{q}=0\\
    \dv{t} \qty(m\vb{v}-\frac qc\vb{A}(\vb{r},t))-q\nabla\phi(\vb{r})+\frac qc\nabla\qty(
    \vb{v}\cdot\vb{A}(\vb{r},t))=0\\
    m\ddot{\vb{r}}-\frac qc\dv{\vb{A}(\vb{r},t)}{t}+\pdv{\vb{A}}{t}=-q\vb{E}-\frac qc\nabla(\vb{v}\cdot
    \vb{A})\\
\dv{\vb{A}}{t}=\pdv{\vb{A}}{t}+\vb{v}(\nabla\cdot\vb{A})\\
m\ddot{\vb{r}}=-q\vb{E}-\frac qc\qty(\nabla(\vb{v}\cdot\vb{A})-\vb{v}(\nabla\cdot\vb{A}))
\end{align}

Qui dit écrire un lagrangien, on peut maintenant travailler sur un Hamiltonien.
Commençons par définir le moment canonique

\begin{align}
    \vb{p}&=m\vb{v}+\frac qc\vb{A}\\
    \mathscr{H}&=\vb{p}\cdot\vb{v}-\mathscr{L}\\
    \vb{v}&=\frac1m\vb{p}-\frac qc\vb{A}\\
    \mathscr{H}&=\frac1m\vb{p}\cdot(\vb{p}-\frac qc\vb{A})-\qty(\frac12\frac m{m^2}
    \qty(\vb{p}-\frac qc\vb{A})^2-q\phi+\frac qc(\vb{p}-\frac qc\vb{A})\cdot\vb{A})\\
               &=\frac{\qty(\vb{p}-\frac qc\vb{A})^2}{2m}+q\phi\\
               &=\frac{p^2}{2m}-\frac q{2mc}(\vb{p}\cdot\vb{A}+\vb{A}\cdot\vb{p})
               +\frac{q^2}{2mc^2}A^2+q\phi,
\end{align}
ce qui représente l'électrodynamique classique. Avec un chamo magnétique constant,
on peut utiliser la jauge symétrique

\begin{align}
    \vb{A}&=-\frac12\vb{r}\times\vb{B}.
\end{align}

On se justifie par

\begin{align}
    \nabla\times\vb{A}&=\frac12\nabla\times(\vb{r}\times\vb{B})\\
                      &=\frac12\qty((\vb{B}\cdot\nabla)\vb{r}-(\vb{r}\cdot\nabla)\vb{B}
                      +\vb{r}(\nabla\cdot\vb{B})-\vb{B}(\nabla\cdot\vb{r}))\\
                      &=\frac12\qty(\vb{B}-(\vb{r}\cdot\nabla)\vb{B}+0-3\vb{B})\\
                    &=\vb{B}+\frac12(\vb{r}\cdot\nabla)\vb{B},
\end{align}
ce qui représente effectivement un champ magnétique constant. Continuons le
développement du hamiltonien

\begin{align}
    \mathscr{H}_{\rm class.}&=\frac{p^2}{2m}+\frac q{2mc}\vb{p}\cdot(\vb{r}\times\vb{B})
    +\frac{q^2}{2mc^2}A^2+q\phi\\
                            &=\frac{p^2}{2m}-\gamma\vb{L}\cdot\vb{B}+\frac{q^2}{2mc^2}
                            A^2+q\phi,
\end{align}
où l'on voit apparaître le terme de Zeeman et le terme diamagnétique.

% subsection Force de Lorentx (end)

\subsection{Densité de courant lié} % (fold)
\label{sub:Densité de courant lié}

Dans un solide, une boucle de courant présente une densité de moment magnétique

\begin{align}
    \vb{J}_{M}&=c\nabla\times\vb{M}(\vb{r})\\
    \nabla\times\vb{B}&=\frac1c\qty(\pdv{\vb{E}}{t}+4\pi\int\vb{J}(\vb{r})+\vb{J}_M(\vb{r})\dd[3]{r})\\
    \nabla\times\vb{H}&=\frac{4\pi}{c}\vb{J}\\
    \vb{H}&=\vb{B}-4\pi\vb{M}(\vb{r}),
\end{align}
en réponse linéaire, nous avons que
\begin{align}
    \vb{M}&=\chi_m\vb{H}.
\end{align}
Ceci est une approximation, en général, la réponse pourrait ne pas être linéaire.

\begin{align}
    \vb{B}&=\vb{H}\qty(1+4\pi\chi_m)\\
          &=\mu\vb{H}
\end{align}
Les différents régimes possible de $\mu$ sont le diamagnétisme, soit $\mu<1$,
le paramagnétisme $\mu\ge1$ et le ferromagnétisme $\mu\gg1$.

% subsection Densité de courant lié (end)

\subsection{Théorème de Bohr Van-Leuven} % (fold)
\label{sub:Théorème de Bohr Van-Leuven}

Posons un petit dipole magnétique. On cherche sa fonction de partition

\begin{align}
    Z_i&=\int\int\frac{\dd[3]{r_i}\dd[3]{p_i}}{h^3}e^{-\beta \mathscr{H}}\\
       &=\int\int\frac{\dd[3]{r_i}\dd[3]{p_i}}{h^3}e^{-\beta (\vb{p}-\frac qc\vb{A})^2/2m}.
\end{align}
On peut se rendre compte qu'en effectuant le changement de variable $\vb{p'}=\vb{p}-\frac qc\vb{A}$,
l'intégrale devient la même que pour le cas sans champ magnétique. Il est donc
impossible d'avoir des effets magnétiques statistique en utilisant seulement
la physique classique.
\begin{align}
    \expval{\vb{m_i}}&=
       \int\int\frac{\dd[3]{r_i}\dd[3]{p_i'}}{h^3}e^{-\beta (\vb{p'})^2/2m} = 0.
\end{align}
Comme il s'agit d'une fonction paire.

% subsection Théorème de Bohr Van-Leuven (end)

\subsection{Équation de Dirac} % (fold)
\label{sub:Équation de Dirac}

Partons de l'hamiltonien, maintenant quantique

\begin{align}
    \mathscr{H}&=\frac{p^2}{2m}-\gamma\qty(\vb{p}\cdot\vb{A}+\vb{A}\cdot\vb{p})
    +\frac{q^2}{2mc^2}A^2\\
               &=\frac{p^2}{2m}-\gamma\vb{L}\cdot\vb{H}+\frac{q^2}{2mc^2}A^2
\end{align}

L'équation de Schrödinger n'est pas invariante de Lorentz. Ceci est un problème
pour la relativité. Il faut donc modifier l'Hamiltonien de sorte à obtenir
l'équation de Dirac
\begin{align}
    E_{\rm class.}&=\frac{P^2}{2m}\\
    E_{\rm rel.}&=P^2c^2+m^2c^4\\
    i\hbar\pdv{t} \ket{\psi(t)}&=\mathscr{H}\ket{\psi(t)}
\end{align}
On peut donc postuler
\begin{align}
    \mathscr{H}&=c\va{\alpha}\cdot\vb{p}+\beta mc^2\\
    \mathscr{H}^2&=p^2c^2+m^2c^4\\
                 &=c^2\qty(\va{\alpha}\cdot\vb{p})^2\beta^2m^2c^4+
                 c\va{\alpha}\cdot\vb{p}\beta mc^2+\beta mc^2\va{\alpha}\cdot\vb{p}
\end{align}
Calculons le produit scalaire $\va{\alpha}\cdot\vb{p}$
\begin{align}
    \va{\alpha}\cdot\vb{p}&=\qty(\alpha_ip_i)^2\\
                          &=\alpha_ip_i\alpha_jp_j\\
                          &=\alpha_i^2p_i^2+\sum_{i\neq j}\{\alpha_i,\alpha_j\}
                          p_ip_j
\end{align}
On doit poser $\{\alpha_i,\alpha_i\}=0$, ce qui implique que les symboles
$\alpha_i$ anti-commutent. On peut donc continuer
\begin{align}
    mc^2\{\alpha_i,\beta\}p_i=0\\
    \Rightarrow\ \{\alpha_i,\beta\}=0.
\end{align}
On peut trouver des matrices qui respectent ces conditions. Les matrices de Dirac
sont
\begin{align}
    \beta&=
    \begin{pmatrix}
        1 & 0 \\ 0 & -1
    \end{pmatrix}\\
    \va{\alpha}&=
    \begin{pmatrix}
        0 & \va{\sigma}\\
        \va{\sigma} & 0
    \end{pmatrix}
\end{align}
l'équation de Schrödinger se transforme donc en
\begin{align}
    i\hbar\pdv{t} \ket{\psi}&=\qty(c\va{\alpha}\cdot\vb{p}+\beta mc^2)\ket{\psi}\\
    E\psi(\vb{r})&=\qty(c\va{\alpha}\cdot\vb{p}+\beta mc^2)\psi(\vb{r})\\
    \psi(\vb{r})&=
    \begin{pmatrix}
        \chi(\vb{r})\\ \phi(\vb{r})
    \end{pmatrix}\\
    E \begin{pmatrix}
        \chi \\ \phi
    \end{pmatrix} &= \qty(
    \begin{pmatrix}
        0 & c\va{\sigma}\cdot\vb{p}\\
        c\va{\alpha}\cdot\vb{p} & 0
    \end{pmatrix}+
    \begin{pmatrix}
        mc^2 & 0\\ 0 & -mc^2
    \end{pmatrix}
    )\begin{pmatrix}
        \chi\\\phi
    \end{pmatrix}\\
                \begin{pmatrix}
                0 \\ 0
            \end{pmatrix}&=
    \begin{pmatrix}
        E-mc^2 & -c\va{\sigma}\cdot\vb{p}\\
        -c\va{\sigma}\cdot\vb{p} & E+mc^2
    \end{pmatrix}
    \begin{pmatrix}
        \chi \\\phi
    \end{pmatrix}
\end{align}
Lorsqu'on utilise le couplage minimal, il faut faire la substitution
$\vb{p}\rightarrow\va{\pi}=\vb{p}-\frac qc\vb{A}$.
On souhaite retrouver l'équation de Schrödinger.
\begin{align}
    (E-mc^2)\chi-c\va{\sigma}\cdot\va{\pi}\phi&=0\\
    (-c\va{\sigma}\cdot\va{\pi})\chi+(E+mc^2)\phi&=0\\
    \Rightarrow\ \phi&=\frac{c\va{\sigma}\cdot\va{\pi}}{E_s+2mc^2}\chi
\end{align}
Où $E_s=E-mc^2$. Dans la limite non-relativiste, $E_s\sim0$, donc
\begin{align}
    \phi&\simeq\frac{c\va{\sigma}\cdot\va{\pi}}{2mc^2}\chi\\
    E_s\chi-c\va{\sigma}\cdot\va{\pi}\qty(\frac{c\va{\sigma}\cdot\va{\pi}}{2mc^2}\chi)&=0\\
    E_s\chi&=\frac{\qty(\va{\sigma}\cdot\va{\pi})^2}{2m}\chi
\end{align}
qui est une équation à la Schrödinger.
\begin{align}
    \va{\pi}&=\vb{p}\\
    \qty(\va{\sigma}\cdot\vb{A})\qty(\va{\sigma}\cdot\vb{B})
            &=\vb{A}\cdot\vb{B}+i\qty(\vb{A}\times\vb{B})\cdot\va{\sigma}\\
    \qty(\va{\sigma}\cdot\vb{p})^2&=p^2
\end{align}
On ajoute maintenant un potentiel
\begin{align}
    \mathscr{H}&=c\va{\alpha}\cdot\vb{p}+\beta mc^2+V(\vb{r})\\
    \mathscr{H}&=
    \begin{pmatrix}
        E-V-mc^2 & -c\va{\sigma}\cdot\vb{p}\\
        -c\va{\sigma}\cdot\vb{p} & E-V+mc^2\\
    \end{pmatrix}\\
    -c&\va{\sigma}\cdot\vb{p}\chi+\qty(E-V+mc^2)\phi=0\\
    \phi&=\frac{1}{E_s-V+2mc^2}c\va{\sigma}\cdot\vb{p}\chi
\end{align}
La limite non-relativiste est donc maintenant
\begin{align}
    \phi=\frac1{2mc^2}\qty(1+\frac{E_s-V}{2mc^2})c\va{\sigma}\cdot\vb{p}\chi
    \qty(E_s-V)\chi-c\va{\sigma}\cdot\vb{p}\phi=0\\
    (E_s-V)\chi=c\va{\sigma}\cdot\vb{p}\qty(\frac1{2mc^2}\qty(1-\frac{E-V}{2mc^2}
    c\va{\sigma}\cdot\vb{p}\chi))\\
    E_s\chi=\qty(\frac{c\va{\sigma}\cdot\vb{p}}{2mc^2}\qty(1-\frac{E_s-V}{2mc^2})
    c\va{\sigma}\cdot\vb{p}+V)\chi\\
    E_s=\qty(\qty(\frac{p^2}{2m}+V)-\frac1{2m}(\va{\sigma}\cdot\vb{p})(E_s-V)(\va{\sigma}
    \cdot\vb{p}))\chi.
\end{align}
On veut simplifier le second terme.
\begin{align}
    [V(\vb{r}),\vb{p}]&=V\vb{p}-\vb{p}V\\
    (E_s-V)(\va{\sigma}\cdot\vb{p})
                      &=\va{\sigma}\cdot\qty((E_s-V)\vb{p})\\
                      &=\va{\sigma}\cdot\qty(\vb{p}(E_s-V)-[V,\vb{p}])\\
    E_s\chi&=H_0-\frac1{2m}(\va{\sigma}\cdot\vb{p})\qty((\va{\sigma}\cdot\vb{p})
    \frac{E_s-V}{2mc^2}-\frac1{2mc^2}\va{\sigma}\cdot[V,\vb{p}]
    )\\
           &=H_0-\frac1{2m}\frac{p^2}{2mc^2}\frac{p^2}{2m}
           +\frac1{4m^2c^2}(\va{\sigma}\cdot\vb{p})\qty(\va{\sigma}\cdot[V,\vb{p}]) \\
           &=H_0-\frac{p^4}{8m^3c^2}
           -\frac{\vb{p}\cdot[V,\vb{p}]+i\va{\sigma}\cdot(\vb{p}\times[V,\vb{p}])}
           {4m^2c^2}\\
    [\vb{p},V(\vb{r}]&=-i\hbar\nabla V(\vb{r}=-i\hbar e^2\frac{\vb{r}}{r^3}\\
    -i\frac{i}{4m^3c^2}\va{\sigma}\cdot\qty(\vb{p}\times\qty(-i\hbar e^2
    \frac{\vb{r}}{r^3}))
           &=\frac{e^2}{2m^3c^2r^3}\vb{L}\cdot\vb{S}
\end{align}
Ce terme est le couplage spin-orbite.
\begin{align}
    W_D&=\frac{-\vb{p}}{4m^2c^2}\cdot[\vb{p},V]\\
    W_D^\dagger&=\frac{[V^\dagger,\vb{p}^\dagger]\cdot\vb{p}^\dagger}{4m^2c^2}\\
    W_D\neq W_D^\dagger
\end{align}
Ce terme nE'st pas hermitien, ce qui est un problème. Ceci vient du fait que
nous avons pris pour acquis que la composante $\phi$ du spineur de Dirac est
petite, ce qui peut ne pas être vrai.
\begin{align}
    \int\qty(\abs{\phi}^2+\abs{\chi}^2)\dd[3]{r}=1\\
    \int\qty(\chi^*\frac{p^2}{4m^2c^2}\chi+\chi^*\chi)\dd[3]{r}\\
    \int\qty(\qty(1+\frac{p^2}{8m^2c^2})\chi)^\dagger
    \qty(1+\frac{p^2}{8m^2c^2})\chi\dd[3]{r}=1\\
    \int\abs{\chi_s}^2\dd[3]{r}
\end{align}
On peut donc réécrire notre équation avec $\chi_s$ maintenant,
\begin{align}
    E_s\chi&=E_s\frac{\chi_s}{\qty(1+\frac{p^2}{8m^2c^2})}\\
           &=\mathscr{H}_{\rm eff.}\chi\\
           E_s\frac{\chi_s}{1+\frac{p^2}{8m^2c^2}}
           &=\mathscr{H}\frac{\chi_s}{1+\frac{p^2}{8m^2c^2}}\\
           &\simeq\mathscr{H}_{\rm eff.}\chi_s+
           \qty[\frac{p^2}{8m^2c^2},\mathscr{H}_{\rm eff.}]\chi_s\\
           &=\mathscr{H}_{\rm eff.}\chi_s
           +\frac1{8m^2c^2}[p^2,V]\chi_s.
\end{align}
Retournons au $W_D$,
\begin{align}
    W_D'&=W_D+H_{\rm corr.}\\
        &=\frac{[p^2,V]}{8m^2c^2}
        -\frac{1}{4m^2c^2}\vb{p}\cdot[\vb{p},V]\\
        &=\frac1{8m^2c^2}\qty(
        -\vb{p}\cdot[\vb{p},V]+[\vb{p},V]\cdot\vb{p})\\
        &=\frac1{8m^2c^2}[[\vb{p},V]\vb{p}]\\
        &=\frac{(i\hbar)^2}{8m^2c^2}\nabla^2V\\
        V=-\frac{e^2}r\\
        \nabla^2V=-4\pi e^2\delta(\vb{r})\\
    W_D&=\frac{\pi\hbar e^2}{2m^2c^2}\delta(\vb{r}).
\end{align}
On nomme ce terme le terme de Darwin. Cette petite correction d'énergie intervient
pour les orbitales $s$, comme elle est la seule orbitale qui a un poid non nul
à $\vb{r}=0$. Avec tous ces développement, on peut maintenant écrire le
Hamiltonien de structure fine
\begin{align}
    \mathscr{H}_{\rm sf.}&=H_0+H_{mv}+H_{so}+H_D\\
                         &=H_0+\frac{p^4}{8m^3c^4}+\frac{e^2}{2m^2c^2r^3}\vb{L}
                         \cdot\vb{S}+\frac{\pi e^2\hbar^2}{2m^2c^2}\delta(\vb{r})
\end{align}
% subsection Équation de Dirac (end)

\subsection{Magnétisme atomique} % (fold)
\label{sub:Magnétisme atomique}

Dans un atome, il y a $Z$ électrons.
\begin{align}
    \mathscr{H}&=H_0+\sum_{i=1}^Z\qty(-\gamma(\vb{L}_i\times\vb{S}_i)\cdot\vb{H})
    +\lambda(r_i)\vb{L}_i\cdot\vb{S}_i
    +\frac{e^2}{8mc^2}\qty(\vb{H}\times\vb{R}_i)^2
\end{align}
Pour certaines molécules, seulement le terme diamagnétique reste, comme il ne
dépend ni de $\vb{L}$, ni de $\vb{S}$. Prenons un champ magnétique constant
\begin{align}
    \mathscr{H}&=H_0+\sum_{i=1}^Z\frac{e^2}{8mc^2}(\vb{H}\times\vb{R}_i)^2\\
    W&=\frac{e^2H^2}{8mc^2}\sum_{i=1}^Z(x^2+y^2)\\
    \Delta E_0&=\expval{0|W|0}\\
              &=\frac{e^2H^2}{8mc^2}\sum_{i=1}^Z\expval{0|x^2+y^2|0}\\
    \expval{0|x^2|0}=\expval{0|y^2|0}&=\frac13\expval{0|R_i^2|0}\\
\Delta E_0&=\frac{e^2H^2}{12mc^2}\sum_{i=1}^{Z_{\rm eff.}}\expval{0|R_i^2|0}\\
m_d&=\frac{\delta\Delta E_0}{\delta H}=\chi_d H\\
\chi_d&=-\frac{e^2}{4mc^2}\sum_{i=1}^{Z_{\rm eff.}}\expval{0|R_i^2|0}
\end{align}

% subsection Magnétisme atomique (end)

\subsection{Règles de Hund} % (fold)
\label{sub:Règle de Hund}

Il y a trois règles qu'il faut respecter pour remplir les couches électroniques
\begin{enumerate}
    \item Il faut maximiser $\vb{S}=\sum_i\vb{S}_i$.
    \item Il faut maximiser $\vb{L}=\sum_i\vb{L}_i$.
    \item Partant du terme spin orbite, $W_{so}=\sum_i\lambda_i\vb{L}_i\cdot\vb{S}_i
        =\widetilde{\lambda}\vb{L}\cdot\vb{S}$, avec $\widetilde{\lambda}>0$ si
        plus que demi-rempli. Il faut donc minimiser $J$ si nous sommes à plus
        que demi-rempli, et maximiser $J$ lorsqu'on est à moins que demi-rempli.
\end{enumerate}
\begin{align}
    W_{so}&=\widetilde{\lambda}\vb{L}\cdot\vb{S}\\
    \Delta E_0&=\widetilde{\lambda}\expval{
    J,M,S|\vb{L}\cdot\vb{S}|J,M,S}
\end{align}
Rappel
\begin{align}
    J_i^2\ket{j_1,j_2,m_1,m_2}&=j_i(j_1+1)\hbar^2\ket{j_1,j_2,m_1,m_2}\\
    J_{i,z}\ket{j_1,j_2,m_1,m_2}&=m_i\hbar\ket{j_1,j_2,m_1,m_2}\\
    \vb{J}&=\vb{J}_1+\vb{J}_2\\
    J^2\ket{j,m}&=j(j+1)\hbar^2\ket{j,m}\\
    J_z&=m\hbar\ket{j,m}\\
    J_{1\pm}\ket{j_1,j_2,m_1,m_2}&=\hbar\sqrt{j_1(j_1+1)-m_1(m_1\pm1)}
    \ket{j_1,j_2,m_1\pm1,m_2}\\
    J_{\pm}\ket{j,m}&=\hbar\sqrt{j(j+1)-m(m\pm1)}
    \ket{j,j,m\pm1}\\
    \ket{j,m}&=\ket{j_1+j_2,j_1+j_2-1}\\
             &=\sqrt{\frac{j_1}{j_1+j_2}}\ket{j_1,j_2;j_1-1,j_2}
             +\sqrt{\frac{j_2}{j_1+j_2}}\ket{j_1,j_2;j_1,j_2-1}
\end{align}
Les bons nombre quantiques sont donc $j, m,j_1,j_2$.

% subsection Règle de Hund (end)


