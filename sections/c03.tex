\section{Introduction aux Transitions de Phase} % (fold)
\label{sec:Introduction aux Transitions de Phase}

Il existe plusieurs sortes de transitions de phases. Selon la classification
de Erhenfest, les transitions aux premier ordre et au second ordre.
\begin{enumerate}
    \item Au premier ordre, la première dérivée de l'énergie est discontinue.
        $Q\neq0$.
    \item Au second ordre, la seconde dérivée de l'énergie est discontinue.
        $Q=0$.
\end{enumerate}
La quantité la plus appropriée pour traiter des transitions de phase est
l'énergie libre de Gibbs. On peut donc égaler les potentiels chimiques pour
une transition du premier ordre
\begin{align}
    \mu_1(T,P)&=\mu_2(T,P)\\
    \mu_1(T+\delta T,P+\delta P)&=
    \mu_2(T+\delta T,P+\delta P)\\
    \qty[\pdv{\mu_1}{T}_P-\pdv{\mu_2}{T}_P]\dd{T}&=
    \dd{p}\qty[\pdv{\mu_2}{p}_T-\pdv{\mu_1}{p}_T]
\end{align}
On peut calculer la dérivée du potentiel chimique selon la température à
pression constante. Partons de la relation de Gibbs-Duhem
\begin{align}
    -S\dd{T}+V\dd{p}-N\dd{\mu}=0\\
    \qty(\pdv{\mu}{T})_p=\frac{S}{N}=s
\end{align}
Ceci nous mène à la relation de Clausius-Clapeyron
\begin{align}
    -(s_1-s_2)\dd{T}=(v_2-v_1)\dd{p}\\
    \dv{p}{T}=\frac{s_1-s_2}{v_1-v_2}
\end{align}



% section Introduction aux Transitions de Phase (end)

\subsection{Comportement critique} % (fold)
\label{sub:Comportement critique}

Pour l'eau
\begin{align}
    \kappa&=-\frac1V \qty(\pdv{V}{p})_T>0\\
    \kappa(T)\sim (T-T_c)^{-\gamma}\qquad(T>T-c)\\
    \kappa(T)\sim (T_c-T)^{-\gamma'}\qquad(T<T-c)
\end{align}

On introduit le paramètre d'ordre
\begin{align}
    \rho_L-\rho_G\sim(T_c-T)^\beta
\end{align}

Pour le magnétisme, on utilise la susceptibilité magnétique
\begin{align}
    \chi=\qty(\pdv{M}{H})_{T,H\rightarrow0}\\
    \chi\sim (T-T_c)^{-\gamma}\qquad(T>T_c)\\
    \chi\sim (T_c-T)^{-\gamma'}\qquad(T<T_c)
\end{align}

On introduit le paramètre d'ordre
\begin{align}
    M=M|_{H=0}+\chi H\\
    M(T,H)=M(T,0)+\chi(T)H\\
    M\sim (T_c-T)^\beta\qquad(T<T_c)
\end{align}
Les $\beta$ expérimentaux sont près de $0.35$ dans les deux cas, ce qui indique
une similarité dans les concepts. On peut aussi parler de chaleur spécifique

\begin{align}
    C_v=\qty(\pdv{Q}{T})_{V,N}\\
    =T\qty(\pdv{S}{T}­)_V\\
    C_V\sim(T-T_c)^\alpha\qquad(T>T_c)\\
    C_V\sim(T_c-T)^{\alpha'}\qquad(T<T_c)\\
\end{align}

% subsection Comportement critique (end)
