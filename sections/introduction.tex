% Removing titlerule from section: Introduction
\titleformat{\section}{
\normalfont\Large\bfseries{
  \color{\MasterColor}\titlerule[0.0pt]
  }
}{\thesection}{1em}{}

\section*{Magnétisme} \label{sec: intro}

On commence avec la loi de Biot-Savard:
\begin{align}
    \vb{B}(\vb{r})&=\frac1c \int\vb{J}(\vb{r})\times\frac{\vb{r}-\vb{r}'}{\norm{\vb{r}-\vb{r}'}^3}\dd[3]{r'}
\end{align}
On peut calculer que

\begin{align}
    \frac{\vb{r}-\vb{r'}}{\norm{\vb{r}-\vb{r}'}^3}&=-\nabla\qty(\frac{1}{\norm{\vb{r}-\vb{r'}}}).
\end{align}
Nous allons utiliser la notation que $\nabla$ représente la dérivée par rapport
au vecteur $\vb{r}$, nous pourrions aussi utiliser la notation $\nabla_{\vb{r}}$.
Lorsque nous voulons utiliser la dérivée par rapport au vecteur $\vb{r'}$, nous
utilisons $\nabla'$.
\begin{align}
    \vb{B}(\vb{r})&=\frac1c\nabla\times\int\frac{\vb{J}(\vb{r'})}{\norm{\vb{r}-\vb{r'}}}\dd[3]{r'}.
\end{align}
On obtient cette équation grâce aux propriétés du gradient:
\begin{align}
    \nabla\times(f\vb{J})&=\nabla f\times\vb{J}+f\nabla\times\vb{J},
\end{align}
en prenant $f=\frac1{\norm{\vb{r}-\vb{r'}}}$.
\begin{align}
    \nabla\times(f\vb{J})&=\nabla\qty(\frac1{\norm{\vb{r}-\vb{r'}}})\times\vb{J}
        +\frac1{\norm{\vb{r}-\vb{r'}}}\nabla\times\vb{J}(\vb{r'})
\end{align}
Le rotationnel d'une quantité ne dépendant pas explicitement de $\vb{r}$ est
le vecteur nul. On voit donc émerger le potentiel vecteur:
\begin{align}
    \vb{B}&=\nabla\times\vb{A}\\
    \vb{A}&=\frac1c\int\frac{\vb{J}(\vb{r'})}{\norm{\vb{r}-\vb{r'}}}\dd[3]{r'}.
\end{align}
Lorsqu'on parle de potentiel vecteur, il est important de bien comprendre la notion
de choix de jauge. On peut se faire une image de ce que le choix de jauge est
en pensant au choix du zéro de l'énergie potentielle. Le potentiel vecteur est
lié à la densité de courant pondérée par la distance.

\subsection{Unités CGS} % (fold)
\label{sub:Unités CGS}
Les unités CGS sont définies avec le centimètre, le gramme et la seconde comme
unités de base, ce qui est différent de SI où l'on utilise le mètre, le kilogramme
et la seconde. Les équations de Maxwell en SI sont

\begin{align}
    \nabla\cdot\vb{B}&=0\\
    \nabla\cdot\vb{E}&=\frac\rho\epsilon_0\\
    \nabla\times\vb{B}&=\mu_0\qty(\epsilon_0\pdv{\vb{E}}{t}+\vb{J})\\
    \nabla\times\vb{E}&=-\pdv{\vb{B}}{t},
\end{align}

En CGS, elles s'écrivent plustot

\begin{align}
    \nabla\cdot\vb{B}&=0\\
    \nabla\cdot\vb{E}&=4\pi\rho\\
    \nabla\times\vb{B}&=\frac1c\qty(\pdv{\vb{E}}{t}+4\pi\vb{J})\\
    \nabla\times\vb{E}&=-\frac1c\pdv{\vb{B}}{t}
\end{align}

% subsection Unités CGS (end)

\subsection{Expension multipolaire} % (fold)
\label{sub:Expension multipolaire}

On peut faire l'expension de

\begin{align}
    \frac1{\norm{\vb{r}-\vb{r'}}}=\frac1{\norm{r}}+\frac{\vb{r}\cdot\vb{r'}}{\norm{\vb{r}}^3}+\cdots
\end{align}
où cette expression est une bonne approximation pour $\norm{\vb{r'}}\ll\norm{\vb{r}}$.
On peut réécrire le potentiel vecteur

\begin{align}
    \vb{A}&=\frac1c\frac1{\norm{r}}\int\vb{J}(\vb{r'})\dd[3]{r}
    +\frac1c\frac1{\norm{r}^3}\int(\vb{r}\cdot\vb{r'})\vb{J}(\vb{r'})\dd[3]{r'}
    +\cdots
\end{align}

On définit donc
\begin{align}
    \vb{A}_{\rm dip.}&=-\frac1c\frac1{\norm{r}^3}\sum_i\hat{n}_i\frac12\qty[
    \vb{r}\times\int(\vb{r'}\times\vb{J})\dd[3]{r'}]_i\\
    \vb{M}(\vb{r}&=\frac1{2c}\vb{r}\times\vb{J}(\vb{r}),
\end{align}
où $\vb{M}$ est la densité de moment magnétique dipolaire.
\begin{align}
    \vb{A}_{\rm dip.}&=-\frac1{\norm{\vb{r}}^3}\vb{r}\times
    \int\vb{M}(\vb{r'})\dd[3]{r'}\\
                     &=-\frac{\hat{r}\times\vb{m}}{r^2},
\end{align}
où $\vb{m}$ est le moment magnétique dipolaire. On peut écrire le champ magnétique
lié au potentiel vecteur dipolaire que nous avons écrit

\begin{align}
    \vb{B}_{\rm dip.}(\vb{r})&=\nabla\times\vb{A}_{\rm dip.}(\vb{r})\\
                             &=\frac{3\hat{r}(\hat{r}\cdot\vb{m})-\vb{m}}{\norm{r}^3}
\end{align}
qui est une propriété à démontrer pour le prochain cours. On peut considérer le
solide comme un ensemble de petites boucles de courant, ce qui ferait émerger
le magnétisme

\begin{align}
    \vb{m}&=\frac{I}{2c}\int\vb{r}\times\dd{\boldsymbol{\ell}}\\
    \vb{J}(\vb{r})&=\sum_iq_i\vb{v}_i\delta(\vb{r}-\vb{r}_i\\
    \vb{m}&=\sum_i\gamma_i\vb{L}_i\\
    \gamma_i&=\frac{q_i}{2M_ir},
\end{align}
qui est l'image classique du magnétisme microscopique.

% subsection Expension multipolaire (end)

\subsection{Force de Lorentz} % (fold)
\label{sub:Force de Lorentx}

La force de Lorentz est

\begin{align}
    \vb{F}&=q\qty(\vb{E}+\frac1c\vb{v}\times\vb{B}).
\end{align}

On peut aussi exprimer le lagrangien non-relativiste

\begin{align}
    \mathscr{L}&=\frac12mv^2-V\\
    V&=q\phi(\vb{r})-\frac qc\vb{v}\cdot\vb{A}(\vb{r},t),
\end{align}

où $\phi(\vb{r})$ est le potentiel électrostatique tel que $\vb{E}=-\nabla\phi$.
On peut montrer que les équations différentielles sont équivalentes en  utilisant
les équations d'Euler-Lagrange.

\begin{align}
    \dv{t} \qty(\pdv{\mathscr{L}}{\dot{q_i}})-\pdv{\mathscr{L}}{q}=0\\
    \dv{t} \qty(m\vb{v}-\frac qc\vb{A}(\vb{r},t))-q\nabla\phi(\vb{r})+\frac qc\nabla\qty(
    \vb{v}\cdot\vb{A}(\vb{r},t))=0\\
    m\ddot{\vb{r}}-\frac qc\dv{\vb{A}(\vb{r},t)}{t}+\pdv{\vb{A}}{t}=-q\vb{E}-\frac qc\nabla(\vb{v}\cdot
    \vb{A})\\
\dv{\vb{A}}{t}=\pdv{\vb{A}}{t}+\vb{v}(\nabla\cdot\vb{A})\\
m\ddot{\vb{r}}=-q\vb{E}-\frac qc\qty(\nabla(\vb{v}\cdot\vb{A})-\vb{v}(\nabla\cdot\vb{A}))
\end{align}

Qui dit écrire un lagrangien, on peut maintenant travailler sur un Hamiltonien.
Commençons par définir le moment canonique

\begin{align}
    \vb{p}&=m\vb{v}+\frac qc\vb{A}\\
    \mathscr{H}&=\vb{p}\cdot\vb{v}-\mathscr{L}\\
    \vb{v}&=\frac1m\vb{p}-\frac qc\vb{A}\\
    \mathscr{H}&=\frac1m\vb{p}\cdot(\vb{p}-\frac qc\vb{A})-\qty(\frac12\frac m{m^2}
    \qty(\vb{p}-\frac qc\vb{A})^2-q\phi+\frac qc(\vb{p}-\frac qc\vb{A})\cdot\vb{A})\\
               &=\frac{\qty(\vb{p}-\frac qc\vb{A})^2}{2m}+q\phi\\
               &=\frac{p^2}{2m}-\frac q{2mc}(\vb{p}\cdot\vb{A}+\vb{A}\cdot\vb{p})
               +\frac{q^2}{2mc^2}A^2+q\phi,
\end{align}
ce qui représente l'électrodynamique classique. Avec un chamo magnétique constant,
on peut utiliser la jauge symétrique

\begin{align}
    \vb{A}&=-\frac12\vb{r}\times\vb{B}.
\end{align}

On se justifie par

\begin{align}
    \nabla\times\vb{A}&=\frac12\nabla\times(\vb{r}\times\vb{B})\\
                      &=\frac12\qty((\vb{B}\cdot\nabla)\vb{r}-(\vb{r}\cdot\nabla)\vb{B}
                      +\vb{r}(\nabla\cdot\vb{B})-\vb{B}(\nabla\cdot\vb{r}))\\
                      &=\frac12\qty(\vb{B}-(\vb{r}\cdot\nabla)\vb{B}+0-3\vb{B})\\
                    &=\vb{B}+\frac12(\vb{r}\cdot\nabla)\vb{B},
\end{align}
ce qui représente effectivement un champ magnétique constant. Continuons le
développement du hamiltonien

\begin{align}
    \mathscr{H}_{\rm class.}&=\frac{p^2}{2m}+\frac q{2mc}\vb{p}\cdot(\vb{r}\times\vb{B})
    +\frac{q^2}{2mc^2}A^2+q\phi\\
                            &=\frac{p^2}{2m}-\gamma\vb{L}\cdot\vb{B}+\frac{q^2}{2mc^2}
                            A^2+q\phi,
\end{align}
où l'on voit apparaître le terme de Zeeman et le terme diamagnétique.

% subsection Force de Lorentx (end)

\subsection{Densité de courant lié} % (fold)
\label{sub:Densité de courant lié}

Dans un solide, une boucle de courant présente une densité de moment magnétique

\begin{align}
    \vb{J}_{M}&=c\nabla\times\vb{M}(\vb{r})\\
    \nabla\times\vb{B}&=\frac1c\qty(\pdv{\vb{E}}{t}+4\pi\int\vb{J}(\vb{r})+\vb{J}_M(\vb{r})\dd[3]{r})\\
    \nabla\times\vb{H}&=\frac{4\pi}{c}\vb{J}\\
    \vb{H}&=\vb{B}-4\pi\vb{M}(\vb{r}),
\end{align}
en réponse linéaire, nous avons que
\begin{align}
    \vb{M}&=\chi_m\vb{H}.
\end{align}
Ceci est une approximation, en général, la réponse pourrait ne pas être linéaire.

\begin{align}
    \vb{B}&=\vb{H}\qty(1+4\pi\chi_m)\\
          &=\mu\vb{H}
\end{align}
Les différents régimes possible de $\mu$ sont le diamagnétisme, soit $\mu<1$,
le paramagnétisme $\mu\ge1$ et le ferromagnétisme $\mu\gg1$.

% subsection Densité de courant lié (end)

\subsection{Théorème de Bohr Van-Leuven} % (fold)
\label{sub:Théorème de Bohr Van-Leuven}

Posons un petit dipole magnétique. On cherche sa fonction de partition

\begin{align}
    Z_i&=\int\int\frac{\dd[3]{r_i}\dd[3]{p_i}}{h^3}e^{-\beta \mathscr{H}}\\
       &=\int\int\frac{\dd[3]{r_i}\dd[3]{p_i}}{h^3}e^{-\beta (\vb{p}-\frac qc\vb{A})^2/2m}.
\end{align}
On peut se rendre compte qu'en effectuant le changement de variable $\vb{p'}=\vb{p}-\frac qc\vb{A}$,
l'intégrale devient la même que pour le cas sans champ magnétique. Il est donc
impossible d'avoir des effets magnétiques statistique en utilisant seulement
la physique classique.
\begin{align}
    \expval{\vb{m_i}}&=
       \int\int\frac{\dd[3]{r_i}\dd[3]{p_i'}}{h^3}e^{-\beta (\vb{p'})^2/2m} = 0.
\end{align}
Comme il s'agit d'une fonction paire.

% subsection Théorème de Bohr Van-Leuven (end)

\subsection{Équation de Dirac} % (fold)
\label{sub:Équation de Dirac}

Partons de l'hamiltonien, maintenant quantique

\begin{align}
    \mathscr{H}&=\frac{p^2}{2m}-\gamma\qty(\vb{p}\cdot\vb{A}+\vb{A}\cdot\vb{p})
    +\frac{q^2}{2mc^2}A^2\\
               &=\frac{p^2}{2m}-\gamma\vb{L}\cdot\vb{H}+\frac{q^2}{2mc^2}A^2
\end{align}

% subsection Équation de Dirac (end)


