\section{Ensembles Statistiques} % (fold)
\label{sec:Ensembles Statistiques}

On définit la matrice densité
\begin{align}
\rho&=\sum_\lambda q_\lambda\ket{\psi_\lambda}\bra{\psi_\lambda},
\end{align}
pour $q_\lambda\in[0,1]$. À partir de ça, on peut avoir la probabilité
\begin{align}
    \mathcal{P}_\lambda(a_\lambda)&=\abs{\braket{\psi_\lambda}{a_\lambda}}^2\\
    \mathcal{P}(a_\lambda)&=\sum_\lambda q_\lambda\mathcal_\lambda(a_\lambda).
\end{align}
Grâce à la matrice densité, on peut représenter les mélanges statistiques d'états
quantiques. On peut utiliser ce formalisme pour écrire les valeurs moyennes.
\begin{align}
    \expval{U}&=\sum_a u_a\mathcal{P}_a\\
              &=\sum_a u_a\sum_\lambda q_\lambda\braket{\psi_\lambda}{u_a}
\braket{u_a}{\psi_\lambda}\\
              &=\sum_\lambda q_\lambda\bra{\psi_\lambda}\qty(\sum_a u_a\ket{u_a}
\bra{u_a})\ket{\psi_\lambda}\\
              &=\sum_\lambda q_\lambda\bra{\psi_\lambda}U\ket{\psi_\lambda}\\
              &=\sum_{\lambda,k} q_\lambda\bra{\psi_\lambda}U\ket{u_k}\braket{u_k}{\psi_\lambda}\\
              &=\sum_{\lambda,k} q_\lambda\braket{u_k}{\psi_\lambda}
              \bra{\psi_\lambda}U\ket{u_k}\\
              &=\sum_{k} \bra{u_k}
              \rho U\ket{u_k}\\
              &=\trace[\rho U]
\end{align}
L'opérateur densité a plusieurs propriétés.
\begin{itemize}
    \item Hermiticité
        \begin{align}
            \rho=\rho^\dagger
        \end{align}
    \item Normalisation
        \begin{align}
            \trace[\rho]&=1
        \end{align}
    \item Semi-positivité
        \begin{align}
            \rho\succ0
        \end{align}
    \item Valeurs propres
        \begin{align}
            \sum_i d_i=1
        \end{align}
    \item Évolution temporelle
        \begin{align}
            i\hbar\dv{\rho}{t}&=\qty[H,\rho]\\
        U(t)&=e^{\sfrac{-iHt}{\hbar}}\\
        \rho(t)&=\sum_\lambda q_\lambda
        U(t)\ket{\psi_\lambda}\bra{\psi_\lambda}U^\dagger(t)\\
               &=\sum_\lambda q_\lambda\ket{\psi_\lambda(t)}\bra{\psi_\lambda(t)}
        \end{align}
\end{itemize}

\subsection{Opérateur densité réduit} % (fold)
\label{sub:Opérateur densité réduit}

Soit un espace de Hilbert
\begin{align}
    \mathcal{H}&=\mathcal{H}^A\otimes\mathcal{H}^B\\
    \rho&=\rho_a\otimes\rho_b
\end{align}
La valeur moyenne d'un observable est donc
\begin{align}
    \expval{A}&=\expval{A_a\otimes\mathds{1}_b}\\
              &=\trace\qty[\rho\qty(A_a\otimes\mathds{1}_b)]
\end{align}

% subsection Opérateur densité réduit (end)

\subsection{Entropie en mécanique statistique} % (fold)
\label{sub:Entropie en mécanique statistique}

\begin{align}
    S&=-k_B\sum_\lambda q_\lambda\ln q_\lambda\\
     &=-k_B\trace[\rho\ln\rho]
\end{align}
Les propriétés de l'entropie
\begin{itemize}
    \item Positivité
        \begin{align}
            S\geq0
        \end{align}
    \item Maximum
        \begin{align}
            q_\lambda&=\frac1\Omega
        \end{align}
        Dans ce cas là, on retrouve l'ensemble micro-canonique.
    \item Dans le devoir
        \begin{align}
            S\leq-k_B\trace[D\ln D']\qquad\forall D'
        \end{align}
    \item Additivité\\
        Pour des corrélations nulles,
        \begin{align}
            D&=D_1\otimes D_2\\
            \Rightarrow\ S&=S_1+S_2
        \end{align}
    \item Sous-additivité
        \begin{align}
            D&\neq D_1\otimes D_2\\
            \Rightarrow\ S&\leq S_1+S_2
        \end{align}
    \item Concavité (Devoir)
        \begin{align}
            S\qty(\sum_j\mu_jD_j)\geq\sum_j\mu_jS(D_j)
        \end{align}
\end{itemize}

% subsection Entropie en mécanique statistique (end)
\subsection{Ensembles} % (fold)
\label{sub:Ensembles}

\subsubsection{micro-canonique} % (fold)
\label{sec:micro-canonique}

\begin{align}
    q_\lambda&=\frac1M\qquad\forall\lambda\\
    S&=-k_B\trace[\rho\ln\rho]=k_B\ln M\\
    \Omega(E)&=\rho(E)\delta E=M
\end{align}
On peut obtenir la température
\begin{align}
    \qty(\pdv{S}{E})_{V,N}&=\frac1T\\
                          &=k_B\pdv{\ln\Omega}{E}\\
    \pdv{\ln\Omega}{E}&=\beta
\end{align}

% subsubsection micro-canonique (end)

% subsection Ensembles (end)

\begin{align}
    F&=S(D)-\sum_i\zeta_i\expval{A_i}-\zeta_i\trace[D]\\
    \delta F&=0\\
    D&=\frac{e^{-\sum_i\zeta_iA_i}}{Z}
\end{align}

L'entropie se trouve à partir de la fonction de partition
\begin{align}
    S&=-k_B\trace[D\ln D]\\
     &=k_B\ln Z+k_B\sum_i\zeta_i\expval{A_i}
\end{align}

\begin{align}
    \dd{S}&=\frac1T\dd{E}+\frac pT\dd{V}-\frac\mu T\dd{N}\\
    \dd{S}&=k_B\sum_i\zeta_i\dd{\expval{A}_i}\\
    \pdv{S}{\expval{A_i}}&=k_B\zeta_i
\end{align}

\subsection{Ensemble canonique}

On peut obtenir l'ensemble canonique à partir de ce que l'on a trouvé pour
l'ensemble micro-canonique.
\begin{align}
    F(T,V,N)&=S(D)-\beta\expval{H}\\
    D&=\frac{e^{-\beta H}}{Z}\\
    S&=k_B\ln Z +k_B\beta E\\
    -k_B\ln Z&=E-TS
\end{align}
Où l'on voit apparaître l'énergie libre de Helmoltz.

\subsection{Ensemble isotherme-isobar}

\begin{align}
    S&=k_B\ln Z+k_B\sum_i\zeta_i\expval{A_i}\\
     &=k_B\ln Z+k_B\beta\expval{H}+k_Bp\beta\expval{V}\\
    TS&=k_BT\ln Z+E+pV\\
    G=-k_BT\ln Z_I&=E+pV-TS
\end{align}
Où l'on voit apparaître l'énergie libre de Gibbs.

\subsection{Ensemble Grand-Canonique} % (fold)
\label{sub:Ensemble Grand-Canonique}

\begin{align}
    TS&=k_B\ln\Xi+k_B\beta E+pV-\mu N
\end{align}

% subsection Ensemble Grand-Canonique (end)

% section Ensembles Statistiques (end)
